% (C) Anders Kofod-Petersen
\documentclass[a4paper]{book}
\usepackage[english]{babel}						% Correct English hyphenation
\usepackage[latin1]{inputenc}						% Allow for non-English letters
\usepackage{graphicx}							% To include graphics
\usepackage{natbib}								% Correct citations
%\usepackage{fancyheadings}						% Nice header and footer
\usepackage[linktocpage,colorlinks]{hyperref}			% PDF hyperlink
\usepackage{geometry} 							% Better geometry
%\usepackage[center]					% For cropping documents

% B5 (uncomment to convert to B5 format)
 \geometry{b5paper}

% Author
% Fill in here, and use commands in the text. 
\newcommand{\thesisAuthor}{Eirik Baug and Andreas Norstein}
\newcommand{\thesisTitle}{The best title in the world}
\newcommand{\thesisType}{master project}
\newcommand{\thesisDate}{spring 2019}

% PDF info
\hypersetup{pdfauthor={\thesisAuthor}}
\hypersetup{pdftitle={\thesisTitle}}
\hypersetup{pdfsubject={\thesisType}}
\hypersetup{linkcolor=black}
\hypersetup{citecolor=black}
\hypersetup{urlcolor=black}

%Fancy headings
%\pagestyle{fancy}
%\pagestyle{fancyplain}
%\renewcommand{\chaptermark}[1]{\markboth{#1}{}}
%\renewcommand{\sectionmark}[1]{\markright{#1}{}}
%\lhead[\fancyplain{}{\thepage}]{\fancyplain{}{\let\uppercase\relax\leftmark}}
%\rhead[\fancyplain{}{\let\uppercase\relax\rightmark}]{\fancyplain{}{\thepage}}
%\chead[\fancyplain{}{}]{\fancyplain{}{}}
%\lfoot[\fancyplain{}{}]{\fancyplain{}{}}
%\cfoot[\fancyplain{}{}]{\fancyplain{}{}}
%\rfoot[\fancyplain{}{}]{\fancyplain{}{}}

% Citation format
\bibliographystyle{apalike}
\bibpunct{[}{]}{;}{a}{,}{,}

\begin{document}

%Title page (This is generate automatically from the commands above)
\begin{titlepage}
\noindent {\large \textbf{\thesisAuthor}}
\vspace{2cm}

\noindent {\Huge \thesisTitle}
\vspace{2cm}

\noindent \thesisType, \thesisDate 
\vspace{2cm}

\noindent Artificial Intelligence Group\\ Department of Computer and Information Science\\ Faculty of Information Technology, Mathematics and Electrical Engineering\\

\vfill
\begin{center}
\includegraphics[width=3cm]{figs/NTNUlogo.pdf}
\end{center}
\end{titlepage}

\thispagestyle{empty}

\cleardoublepage

\frontmatter

\section*{Abstract}

This paper provides a template for writing AI project rapports for either the AI specialisation project; masters "datateknikk" or masters "informatikk". The use of the template is recommended and is written in english as we encourage students to submit their project and masters theses in English. 
The template does not form a compulsory style that you are obliged to use. However, the format and contents are a result of a joint AI group initiative thus providing a common starting point for all AI students. For a given project tuning of the template may still be required. Such tuning might involve moving a chapter to a section or vice versa due to the nature of the project..

The abstract is your sales pitch which encourages people to read your work but unlike sales it should be realistic with respect to the contributions of the work. It should include:
\begin{itemize}
\item the field of research
\item a brief motivation for the work
\item what the research topic is and
\item the research approach(es) applied. 
\item contributions
\end{itemize}

The abstract length should be roughly half a page of text --- without lists, tables or figures.  

\clearpage

\section*{Preface}



\vspace{1cm}

The preface includes the facts - what type of project, where it is conducted, who supervised and any acknowledgements you wish to give. 

\vfill

\hfill \thesisAuthor

\hfill Trondheim, \today

\clearpage

\tableofcontents

\listoffigures

\listoftables

\mainmatter
\chapter{Introduction}
\label{cha:Introduction}

All chapters should begin with an introduction before any sections begin. Further, each sections begins with an introduction before  subsections begin. Chapters with just one section or sections with just one sub-section, should be avoided. Think carefully about chapter and section titles as each title stand alone in the table of contents (without associated text) and should convey meaning for the contents of the chapter or section. 

In all chapters and sections it is important to write clearly and concisely. Avoid repetitions and if needed, refer back to the original discussion or presentation. Each new section, subsection or paragraph should provide the reader with new information and be written in your own words. Avoid direct quotes. If you use direct quotes, unless the quote itself is very significant, you are conveying to the reader that you are unable to express this discussion or fact yourself. Such direct quotes also break the flow of the language (yours to someone else's).   





\section{Background and Motivation}\label{cit}
\label{sec:BackgroundAndMotivation}

Having a template to work from provides a starting point. However, for a given project, a slight variation in the template may be required due to the nature of the given project. Further, the order in which the various chapters and sections will be written will also vary from project to project but will seldom start at the abstract and sequentially follow the chapters of the report. One critical reason for this, is that you need to start writing as early as possible and you will begin to write up where you are currently focusing. However, do not leave the abstract until the end. The abstract is the first thing anyone reads of an article or thesis --- after the title; and thus it is important that it is very well written. Abstracts are hard to write so create revisions throughout the course of your project as your project progresses.  

This introduction to background and motivation should state where this project is situated in the field and what the key driving forces motivating this research are. However, keep this section brief as it is still part of the introduction. The motivation will be further extended in chapter~\ref{T-B}, presenting your complete state-of-the-art. 

Note that this template uses italics to highlight where latin wording is inserted to represent text and the text of the template that we wish to draw your attention to. The italics themself are not an indication that such sections should use italics.  

\section{Goals and Research Questions}
\label{sec:Goals and Research Questions}

A masters is a research project and thus there needs to be a question(s) that need answered. Such questions are often a very important part of the results that come out of the specialisation project. For those following the one year masters project, it is desirable to create such questions as early as possible as the formation of such questions provide both an important driving force for the masters project and provide clarity as to the goals sought. However, one will expect to refine the questions and thus the final path of the masters as work progresses. However any refinements should be conducted with care so as to avoid that the original aims, and previous work are not lost.  
It is always good to have one (or max 2) key questions and perhaps some sub questions. 

\begin{description}
\item[Goal] {\it Lorem ipsum dolor sit amet, consectetur adipiscing elit.}
\end{description}

Your goal/objective should be described in a single sentence. In the text under you can expand on this sentence to clarify what is meant by the short goal description. 
The goal of your work is what you are trying to achieve. This can either be the goal of your actual project or can be a broader goal that you have taken steps towards achieving. Such steps should be expressed in the research questions. 
Note that the goal is seldom to build a system. A system is built to to enable experiments to be conducted. The research question/goal would be the goal that the system is implemented to meet.  


\begin{description}
\item[Research question 1] {\it Lorem ipsum dolor sit amet, consectetur adipiscing elit.}
\end{description}

Each research question provides a sub-goal and these should be precise and clearly stated enabling the reader to match your results to the original goals. They will also form the driving force for the experimental plan. 

\begin{description}
\item[Research question 2] {\it Lorem ipsum dolor sit amet, consectetur adipiscing elit.}
\end{description}

{\it Lorem ipsum dolor sit amet, consectetur adipiscing elit. Nam consequat pulvinar hendrerit. Praesent sit amet elementum ipsum. Praesent id suscipit est. Maecenas gravida pretium magna non }

\section{Research Method}
\label{sec:researchMethod}

What methodology will you apply to address the goals: theoretic/analytic, model/abstraction or design/experiment? This section will describe the research methodology applied and the reason for this choice of research methodology.  

\section{Contributions}
\label{sec:IntroContributions}

The main description of the contributions will come in chapter~\ref{cont} after the results are presented. This section just provides a brief summary of the main contributions of the work. This section can also be left out, leaving all discussions in chapter~\ref{cont}.

The format of this section will generally follow the following format:
{\it
Donec non turpis nec neque egestas faucibus nec id neque. Etiam consectetur, odio vitae gravida tempus, diam velit sagittis turpis, a molestie ligula tellus at nunc. Nam convallis consequat vestibulum. Proin dolor neque, dapibus a pellentesque a, commodo a nibh.}

\begin{enumerate}
\item {\it Lorem ipsum dolor sit amet, consectetur adipiscing elit.}
\item {\it Lorem ipsum dolor sit amet, consectetur adipiscing elit.}
\item {\it Lorem ipsum dolor sit amet, consectetur adipiscing elit.}
\end{enumerate}


\section{Thesis Structure}
\label{sec:thesisStructure}

This section provides the reader with an overview of what is coming in the next chapters. You want to say more than what is explicit in the chapter name, if possible, but still keep the description short and to the point. 
	
\chapter{Background Theory and Motivation}\label{T-B}
\label{cha:TheoryAndBackground}

{\it Lorem ipsum dolor sit amet, consectetur adipiscing elit. Nam consequat pulvinar hendrerit. Praesent sit amet elementum ipsum. Praesent id suscipit est. Maecenas gravida pretium magna non interdum. Donec augue felis, rhoncus quis laoreet sed, gravida nec nisi. Fusce iaculis fermentum elit in suscipit.}


\section{Background Theory}
\label{sec:no1}

The background theory depth and breadth depends on the depth needed to understand your project in the different disciplines that your project crosses.  It is not a place to just write about everything you know that is vaguely connected to your project. The theory is here to help the reader that does not know the theoretical basis of your work so that he/she can gain sufficient understanding to understand your contributions. In particular, the theory section provides an opportunity to introduce terminology that can later be used without disttheoryurbing the text with a definition.  In some cases it will be more appropriate to have a separate section for different theory. However, watch that you don't end up with too short sections. Subsections may also be used to separate different background theory. 

When introducing techniques or results, always reference the source. Be careful to reference the original contributor of a technique and not just someone who happens to use the technique. For relevant results to your work, you would want to look particularly at newer results so that you have referenced the most up-to-date work in your area. If you don't have the source handy when writing, mark the test that a reference is needed and add it later. 

Web pages are not reliable sources --- they might be there one day and removed the next; and thus should be avoided, if possible. A verbal discussion is not a source and should not be referenced or described in the text.  

The bulk of citations in the report will appear in section~\ref{cit}. However, you will often need to introduce some terminology and key citations already in this chapter. 

You can cite a paper in the following manners: 

\begin{itemize}
\item when referring to authors:\\
 \citet{authorson10:_secon_best_paper_in_world} stated something rather nice.
\item to cite indirectly: \\
 Papers should be written nicely \citep{authorson10:_secon_best_paper_in_world}\\
or\\
In \cite{authorson10:_secon_best_paper_in_world}, a less detailed template was presented.
\item To just cite the authors: \\
\citeauthor{authorson10:_secon_best_paper_in_world} wrote a nice paper.
%\item Or just the year: \citeyear{authorson10:_secon_best_paper_in_world}.
%\item You can even cite specific pages: \citet[p. 3]{authorson10:_secon_best_paper_in_world}.
\end{itemize}

\vspace{0.5cm}

\noindent
{\bf Introducing figures:} \\

\begin{figure}[ht]
\begin{center}
\includegraphics[width=0.5\columnwidth]{figs/figure1.pdf}
\caption[Boxes and arrows are nice]{Boxes and arrows are nice (adapted from \citet{authorson10:_secon_best_paper_in_world})}
\label{fig:BoxesAndArrowsAreNice}
\end{center}
\end{figure}

Remember that when you borrow figures you should always credit the original author --- such as Figure \ref{fig:BoxesAndArrowsAreNice} (adapted from \citet{authorson10:_secon_best_paper_in_world}). Also don't just put the figure in and leave it to the author to try to understand what the figure is. The figure should be put in to convey a message and you need to help the author to understand the message intended by explaining the figure in the text. 

\vspace{0.5cm}

\noindent
{\bf Introducing tables in the report: }\\

\begin{table}[htbp]
\begin{center}
\begin{tabular}{|c|c|c|c|c|}\hline\hline
This & is & a & nice & table\\\hline
This & is & a & nice & table\\\hline\hline
\end{tabular}
\caption{Example Table}
\end{center}
\label{tab:ExampleTable}
\end{table}%

As you can see from Table \ref{tab:ExampleTable}, tables are nice. However, again, you need to discuss the contents of the table in the text. You don't need to describe every entry but draw the authors attention to what is important for he/she to glean from the table. 

\section{Structured Literature Review Protocol}

Here you need to include your structured review protocol including search engine, search words, research questions  (for search, not the masters research questions), inclusion createrias and evaluation Criterias. 

\section{Motivation}
\label{sec:no2}

Your motivation can be either application driven or technique/methodology driven. However in both cases, there will be an element of methodology driven due to the research focus of our group and the nature of a masters project.  
What other research has been conducted in this area and how is it related to your work? The text should clearly illustrate why your goals and research questions are important to address. This section is thus where your literate review will be presented. It is important when presenting the review that you present an overview of the motivating elements of the work going on in your field and how these relate to your proposal, rather than a list of contributors and what they have done. This means that you need to extract the key important factors for your work and discuss how others have addressed each of these factors and what the advantages/disadvantages are with such approaches. As you mention other authors, you should reference their work. Note that the reference list reflects the literature you have read and have cited. This will only be a subset of the literature that you have read.

\input{./chapters/chapter3-model.tex}
\input{./chapters/chapter4-experiments-and-results.tex}
\chapter{Evaluation and Conclusion}
\label{cha:evaluationAndConclusion}

{\it Lorem ipsum dolor sit amet, consectetur adipiscing elit. Nam consequat pulvinar hendrerit. Praesent sit amet elementum ipsum. Praesent id suscipit est. Maecenas gravida pretium magna non interdum. Donec augue felis, rhoncus quis laoreet sed, gravida nec nisi. Fusce iaculis fermentum elit in suscipit. }

\section{Evaluation}
\label{sec:Evaluation}

When evaluating your results, avoid drawing grand conclusions, beyond that which your results can infact support. Further, although you may have designed your experiments to answer certain questions, the results may raise other questions in the eyes of the reader. It is important that you study the graphs/tables to look for unusual features/entries and discuss these aswell as discussing the main findings in the results. 

\section{Discussion}
\label{sec:Discussion}

In the discussion it is important to include a discussion of not just the merits of the work conducted but also the limitations. 

\section{Contributions}~\label{cont}
\label{sec:Contributions}

What are the main contributions made to the field and how significant are these contribution.  

\section{Future Work}
\label{sec:futureWork}

Consider where you would like to extend this work. These extensions might either be continuing the ongoing direction or taking a side direction that became obvious during the work. Further, possible solutions to limitations in the work conducted, highlighted in ~\ref{sec:Discussion} may be presented. 



\backmatter

\addcontentsline{toc}{chapter}{Bibliography}
\bibliography{bibtex/bibliography}

\chapter{Appendices}
\label{cha:appendices}


\end{document}
